\documentclass{article}

\usepackage{amsmath}
\usepackage[utf8]{inputenc}
\usepackage[T1]{fontenc}
\usepackage{parskip}
\usepackage{graphicx}
\usepackage{geometry}
\usepackage{epstopdf}
\usepackage{siunitx}
\usepackage{eurosym}
\usepackage{listings}
\usepackage{courier}
\usepackage[shortlabels]{enumitem}

\begin{document}
\section*{C++ Programming project plan - Micro Machines 5}
{\large Juuso Korvuo}\\
{\large Lauri Kurkela}\\
{\large Lauri Nyman}\\
{\large Henry Räbinä}

\subsection*{Overview}

The goal of this project is to create a top-down racing game inspired by Micro Machines. The objective of the game is to finish a specified amount of laps before opponent players. The levels consist of roads and other material types which the players can drive on, and checkpoints which have to be reached in a certain order. In addition, the levels include physical objects (e.g., rocks, railings) with which the players can collide.

The initial plan is to use cars as the vehicle types, and plan the levels accordingly. Additional vehicle types and levels for them may be added if we have enough time. Other possible additions are weapons for disrupting other players, a level editor for easy level creation, AI and online multiplayer.

The game will have a menu screen, consisting of at least a start game button, an options menu and an exit button. Possible options are changing the control scheme of the players and the sound settings (if sounds are implemented).

The driving physics of the vehicles and collision detection will be handled by the Box2D physics engine. The user interface and graphics will be drawn using SFML.

\subsection*{Rudimentary class structure and description}
\subsubsection*{Level}
The level class will contain a matrix of block types, where each element of the matrix defines the material of the road at some coordinates. For example, a 1000x1000 pixel level could be defined by a 100x100 matrix, where each element corresponds to a 10x10 pixel block of the level. This allows easy queries to ask what the material of the ground is at some coordinates, and the vehicle physics can be adjusted accordingly (decrease friction on ice, decrease max speed on mud, etc.). The level will also contain a list of all physics objects in it (objects which use collision mechanics). A file from which the level is loaded will contain the values of the block matrix and the necessary details about the physics objects.

\subsubsection*{PhysicsObject}
The objects in the level which use features from the Box2D library will be physics objects. The purpose of this class is to tie the body of the object created using Box2D to the graphics created using SFML, to allow easy drawing of the objects. Vehicles and static obstacles such as rocks inherit from this class.

\subsubsection*{Vehicle, Car, Boat etc.}
Vehicle is an abstract class which defines some common properties and methods of all vehicles (asking coordinates, for example). The child classes have different values for vehicle type specific constants, such as max speed and friction.

\subsubsection*{StaticObstacle}
Static obstacles are objects which have collision properties and are placed on the level.

\subsubsection*{Player}
The player class will contain information about the player, such as their name, control scheme and currently used vehicle.

\begin{figure}[h]
\centering
\includegraphics[width=\textwidth]{classes.pdf}
\caption{Rudimentary class structure.}
\end{figure}

\subsection*{The group and distribution of roles}
The group will aim to meet at least once every week to discuss progress and plan ahead. We will also communicate using Telegram to discuss urgent matters.

The roles will probably change over the course of the project, but the first tasks are distributed as follows:

Juuso will start working on implementing the vehicle classes and the driving physics.

Lauri K. will create the makefile(s) and structure the project directory.

Lauri N. will start working on the Level class.

Henry will start working on the menus.

\end{document}
